\documentclass[10pt,a4paper]{article}

%\usepackage{verbatim}
%\usepackage{fancyhdr}
%\usepackage{graphicx}
%
%\setlength\textwidth{6.5 in}
%\setlength\oddsidemargin{0in}
%\setlength\evensidemargin{0in}
%\setlength\topmargin{-1.0 in}
%\setlength\footskip{0 in}
%\setlength\textheight{9in}

\begin{document}

\title{LOFAR International Single Station Metadata Definition}
\author{Griffin Foster}
\date{\today}
\maketitle

\begin{abstract}
This is a technical description of the metadata format defined for use with
LOFAR international single station data products including: total correlation
matrices (ACC), beamlet statistics (BST), subband statistics (SST), and single
subband correlation matrices (XST). A python module (\texttt{issformat}) has
been written which implements this definition.
\end{abstract}

\section{Introduction}
\label{sec:intro}

\section{Metadata Key Definitions}

\subsection{Generic Keys}

\begin{itemize}
    \item \textbf{Station:} station string, 5 characters, e.g. SE607, UK608,
    IE613, ...
    \item \textbf{RCUmode:} mode of each RCU, valid values: 1-7. If only one
    entry is used then it is assumed all RCUs are the same mode. Otherwise, an
    entry for each RCU is required.
    \item \textbf{Timestamp:} date and time of the file.
    \item \textbf{HBAElements:} optional, when using the HBA in a non-standard
    mode by disabling elements in the tile, e.g. HBA `All-sky' mode, then this
    key is used to store the setup of each tile. A tile state is encoded in a
    4-digit hexidecimal string. Each hexidecimal character represents a row of
    the tile.
    \item \textbf{Special:} Extra entry to include comments for the observation.
\end{itemize}

\subsection{Total Correlation (ACC) Keys}

\begin{itemize}
    \item \textbf{Integration:} correlation integration length in seconds,
    default: 1.
\end{itemize}

\subsection{Beamlet Statistics (BST) Keys}

\begin{itemize}
    \item \textbf{Integration:} correlation integration length in seconds.
    \item \textbf{Bitmode:} beamlet bit mode, 16, 8, or 4 bit resulting in 244,
    488, 976 possible beamlets respectively.
    \item \textbf{Pol:} polarization of beamlet, X or Y.
    \item \textbf{beamlets:}
    \begin{itemize}
        \item \textbf{ID:} beamlet ID number
        \item \textbf{Pointing:} pointing in given coordinate system (theta,
        phi, coord).
        \item \textbf{Subband:} subband ID.
        \item \textbf{RCUs:} RCUs in the beamlet.
    \end{itemize}

\end{itemize}

\subsection{Subband Statistics (SST) Keys}

\begin{itemize}
    \item \textbf{RCU:} RCU ID.
\end{itemize}

\subsection{Subband Correlation (XST) Keys}

\begin{itemize}
    \item \textbf{Integration:} correlation integration length in seconds.
    \item \textbf{Subband:} subband ID.
\end{itemize}

\section{Examples}

\section{\texttt{issformat} use}

\end{document}             % End of document.
