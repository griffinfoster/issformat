\documentclass[10pt,a4paper]{article}

%\usepackage{verbatim}
%\usepackage{fancyhdr}
%\usepackage{graphicx}
%
%\setlength\textwidth{6.5 in}
%\setlength\oddsidemargin{0in}
%\setlength\evensidemargin{0in}
%\setlength\topmargin{-1.0 in}
%\setlength\footskip{0 in}
%\setlength\textheight{9in}

\begin{document}

\title{LOFAR International Single Station Metadata Definition}
\author{Griffin Foster}
\date{\today}
\maketitle

\begin{abstract}
This is a technical description of the metadata format defined for use with
LOFAR international single station data products including: total correlation
matrices (ACC), beamlet statistics (BST), subband statistics (SST), and single
subband correlation matrices (XST). A python module (\texttt{issformat}) has
been written which implements this definition. Metadata is stored as JSON files.
HDF5 is used to provide a metadata wrapper for raw data files.
\end{abstract}

\tableofcontents

\section{Introduction}
\label{sec:intro}

The statistics files produced by a LOFAR station contain no observational
metadata to describe the array configuration. Only time is encoded in the
standard file name. In order to provide the essential metadata we have defined a
set of keys which can be written to a JSON file as an auxiliary file to the raw
data. It is useful to keep a common naming convention for this file by replacing
the .dat extension with a .json extension. In addition, an HDF5 format is
defined to wrapper the raw data and provide the metadata as attributes. In this
document the metadata is defined, along with the JSON format, HDF5 format, and a
python module which can be used to interface with these file types.

\section{Statistics File Types}

\noindent\textbf{Total Correlation (ACC):} This is a correlation matrix of all
antenna pairs in the configured array, for all subbands. For each integration a
single subband is correlated, cycling through all 512 subbands. Each file
contains a single integration per subband. The binary file is made up of 32-bit
floats (real and imaginary, to make a 64-bit complex number). The data is of the
form (512 subbands $\times$ nantpol $\times$ nantpol), where nantpol is 192 (96
antenna elements, 2 polarizations) for an international station. Note, KAIRA has
only 48 antenna elements. The standard naming convention is
\texttt{YYYYMMDD\_HHMMSS\_acc\_512xNANTPOLxNANTPOL.dat}, e.g.
\texttt{20120611\_124534\_acc\_512x192x192.dat}. See \texttt{acc2npy()} for how
to read in an ACC file to a numpy array.
\\

\noindent\textbf{Beamlet Statistics (BST):} The detected and integrated power
from each station beamlet. This is the most complex of the statistics files.
Depended in the bitmode (4, 8, or 16) there are (976, 488, or 244) beamlets
respectively. Each beamlets is made up of a set of RCUs beamformed in a specific
direction for a giver coordinate system for a single subband. Each file contains
the total number of beamlets for multiple integrations. The binary file is made
up of 32-bit floats. The data is of the form (nintegrations $\times$ nbeamlets).
The standard naming convention is \texttt{YYYYMMDD\_HHMMSS\_bst\_00POL.dat}, e.g.
\texttt{20170217\_111340\_bst\_00X.dat}.  See \texttt{bst2npy()} for how to read in
an BST file to a numpy array.
\\

\noindent\textbf{Subband Statistics (SST):} The subband statistics of a single
RCU, i.e. the integrated power spectrum for an antenna element. Multiple
integrations from the same RCU can be stored in a single file. The binary file
is made up of 32-bit floats. The data is of the form (nintegrations $\times$ 512
subbands). The standard naming convention is
\texttt{YYYYMMDD\_HHMMSS\_sst\_rcuRCUID.dat}, e.g.
\texttt{20140430\_153356\_sst\_rcu024.dat}.  See \texttt{sst2npy()} for how to
read in an SST file to a numpy array.
\\

\noindent\textbf{Subband Correlation (XST):} Similar to an ACC file in content,
contains the correlation matrix for a single subband. Multiple integrations of
the subband can be included in the file. The binary file is made up of 32-bit
floats (real and imaginary, to make a 64-bit complex number). The data is of the
form (nintegrations $\times$ 1 subband $\times$ nantpol $\times$ nantpol), where
nantpol is 192 (96 antenna elements, 2 polarizations) for an international
station. Note, KAIRA has only 48 antenna elements.  The standard naming
convention is \texttt{YYYYMMDD\_HHMMSS\_sbSBID\_xst.dat}, e.g.
\texttt{20170728\_184348\_sb180\_xst.dat}.  See \texttt{xst2npy()} for how to
read in an XST file to a numpy array.
\\

\section{Metadata Key Definitions}

A minimum set of keys have been defined that are necessary to describe most (if
not all) possible observing modes using a LOFAR station. A common set of keys
are used for all statistics data types (Section \ref{sec:common_keys}). There
are also keys specific to file types BST (Section \ref{sec:bst_keys}), SST
(Section \ref{sec:sst_keys}), and XST (Section \ref{sec:xst_keys}).

\subsection{Common Keys}
\label{sec:common_keys}

\begin{itemize}
    \item \textbf{Station:} \texttt{string}, station ID, 5 characters, e.g.
    SE607, UK608, IE613, ...
    \item \textbf{RCUmode:} \texttt{int} or list of \texttt{int}, mode of each
    RCU, valid values: 1-7. If only one entry is used then it is assumed all
    RCUs are the same mode. Otherwise, an entry for each RCU is required.
    \item \textbf{Timestamp:} \texttt{string}, date and time of the file.
    \item \textbf{HBAElements:} \texttt{string}, when using the HBA in a
    non-standard mode by disabling elements in the tile, e.g. HBA `All-sky'
    mode, then this key is used to store the setup of each tile. A tile state is
    encoded in a 4-digit hexadecimal string. Each hexadecimal character
    represents a row of the tile. (optional)
    \item \textbf{Special:} \texttt{string}, Extra entry to include comments for
    the observation. (optional)
    \item \textbf{Rawfile:} \texttt{string}, Filename of the raw data file. This
    should not include any path information.
    (optional but recommended)
    \item \textbf{Integration:} \texttt{int}, integration length in seconds,
    default: 1.
\end{itemize}

\subsection{Total Correlation (ACC) Keys}

\subsection{Beamlet Statistics (BST) Keys}
\label{sec:bst_keys}

\begin{itemize}
    \item \textbf{Bitmode:} \texttt{int}, beamlet bit mode, 16, 8, or 4 bit
    resulting in 244, 488, 976 possible beamlets respectively.
    \item \textbf{Pol:} \texttt{string}, polarization of beamlet, X or Y.
    \item \textbf{beamlets:}
    \begin{itemize}
        \item \textbf{ID:} \texttt{int}, beamlet ID number
        \item \textbf{Pointing:} (\texttt{float}, \texttt{float},
        \texttt{string}), pointing in given coordinate system (theta, phi,
        coord). Coordinates are in radians. Valid coordinate systems: J2000,
        HADEC, AZELGEO, ITRF, B1950, GALACTIC, ECLIPTIC, JUPITER, MARS, MERCURY,
        MOON, NEPTUNE, 'PLUTO, SATURN, SUN, URANUS, VENUS
        \item \textbf{Subband:} \texttt{int}, subband ID.
        \item \textbf{RCUs:} list of \texttt{int}, RCUs in the beamlet, OR `all'
        when using all RCUs.
    \end{itemize}

\end{itemize}

\subsection{Subband Statistics (SST) Keys}
\label{sec:sst_keys}

\begin{itemize}
    \item \textbf{RCU:} \texttt{int}, RCU ID.
\end{itemize}

\subsection{Subband Correlation (XST) Keys}
\label{sec:xst_keys}

\begin{itemize}
    \item \textbf{Subband:} \texttt{int}, subband ID.
\end{itemize}

\section{JSON Definition}

Metadata is stored as a text JSON file following a simple <key> : <value> entry.

\subsection{ACC}

The ACC JSON metadata file format is:

\begin{verbatim}
{
    "datatype": <string>,
    "hbaelements": <string> OR null,
    "integration": <int>,
    "rawfile": <string>,
    "rcumode": <int> or [<int>],
    "special": <string> OR null,
    "station": <string>,
    "timestamp": <string>
}
\end{verbatim}

\noindent An example of this format is:

\begin{verbatim}
{
    "datatype": "ACC",
    "hbaelements": null,
    "integration": 1,
    "rawfile": "20120611_124534_acc_512x192x192.dat",
    "rcumode": 3,
    "special": null,
    "station": "UK608",
    "timestamp": "2012-06-11 12:45:34"
}
\end{verbatim}

\subsection{BST}

The BST JSON metadata file format is:

\begin{verbatim}
{
    "beamlets": {
        <string (beamlet ID)>: {
            "coord": <string>,
            "phi": <float>,
            "rcus": [<int>] OR "all",
            "sb": <int>,
            "theta": <float>
        },
		...
    },
    "bitmode": <int>,
    "datatype": <string>,
    "hbaelements": <string> OR null,
    "integration": <int>,
    "pol": <string>,
    "rawfile": <string>,
    "rcumode": <int> or [<int>],
    "special": <string> OR null,
    "station": <string>,
    "timestamp": <string>
}
\end{verbatim}

\noindent For each beamlet entry a string of the beamlet ID (an integer) is the
unique identifier. An example of this format showing the first three beamlets is:

\begin{verbatim}
{
    "beamlets": {
        "0": {
            "coord": "AZELGEO",
            "phi": 0.831966,
            "rcus": "all",
            "sb": 114,
            "theta": -0.381997
        },
        "1": {
            "coord": "AZELGEO",
            "phi": 0.178252,
            "rcus": "all",
            "sb": 148,
            "theta": -0.297535
        },
        "2": {
            "coord": "AZELGEO",
            "phi": 1.829449,
            "rcus": '0,1,2,4,5,6,7,8,9,11,13,14,15,16,17,18,19,23,24,25,27,
            28,29,30,31,32,36,37,38,39,41,42,44,47,48,49,50,51,52,54,55,56,
            57,59,60,61,62,63,64,65,67,68,69,70,71,72,73,74,75,76,77,78,79,
            80,81,82,83,85,86,89,90,91,93,94,95,96,97,98,100,101,102,103,104,
            105,107,108,109,111,113,114,115,116,118,119,120,122,123,124,125,
            126,127,128,129,130,131,133,134,136,137,138,139,140,141,142,143,
            145,148,149,150,151,153,158,159,160,161,163,164,165,166,167,168,
            169,170,171,172,174,175,176,178,179,180,181,182,183,184,185,186,
            187,188,189,190,191',
            "sb": 259,
            "theta": 1.094793
        }
		...
    },
    "bitmode": 8,
    "datatype": "BST",
    "hbaelements": null,
    "integration": 1,
    "pol": "X",
    "rawfile": "20170217_111340_bst_00X.dat",
    "rcumode": 3,
    "special": null,
    "station": "UK608",
    "timestamp": "2017-02-17 11:13:40"
}
\end{verbatim}

\subsection{SST}

The SST JSON metadata file format is:

\begin{verbatim}
{
    "datatype": <string>,
    "hbaelements": <string> OR null,
    "integration": <int>,
    "rawfile": <string>,
    "rcu": <int>,
    "rcumode": <int> or [<int>],
    "special": <string> OR null,
    "station": <string>,
    "subband": <int>,
    "timestamp": <string>
}
\end{verbatim}

\noindent An example of this format is:

\begin{verbatim}
{
    "datatype": "SST",
    "hbaelements": null,
    "integration": 1,
    "rawfile": "20140430_153356_sst_rcu024.dat",
    "rcu": 24,
    "rcumode": 3,
    "special": null,
    "station": "KAIRA",
    "timestamp": "2014-04-30 15:33:56"
}
\end{verbatim}

\subsection{XST}

The XST JSON metadata file format is:

\begin{verbatim}
{
    "datatype": <string>,
    "hbaelements": <string> OR null,
    "integration": <int>,
    "rawfile": <string>,
    "rcumode": <int> or [<int>],
    "special": <string> OR null,
    "station": <string>,
    "subband": <int>,
    "timestamp": <string>
}
\end{verbatim}

\noindent An example of this format is:

\begin{verbatim}
{
    "datatype": "XST",
    "hbaelements": null,
    "integration": 1,
    "rawfile": "20170728_184348_sb180_xst.dat",
    "rcumode": 3,
    "special": null,
    "station": "IE613",
    "subband": 180,
    "timestamp": "2017-07-28 18:43:48"
}
\end{verbatim}

\section{HDF5 Definition}

An HDF5 definition exists to wrap around the raw binary statistics files. A top
level attribute defines the data type. There is one dataset \texttt{data} which
contains an array of with \texttt{DIMENSION\_LABELS} dimensions. The metadata is
contained in the attributes of the data set.

\noindent ACC follows the following layout:

\begin{verbatim}
{
    { Attributes:
        "CLASS": "ACC",
    }
    { data:
        { Attributes:
            "DIMENSION_LABELS": ('time', 'subband', 'antpol1', 'antpol2'),
            "hbaelements": <string> OR null,
            "integration": <int>,
            "rawfile": <string>,
            "rcumode": <int> or [<int>],
            "special": <string> OR null,
            "station": <string>,
            "timestamp": <string>
        {
    }
}
\end{verbatim}

\noindent BST follows the following layout:

\begin{verbatim}
{
    { Attributes:
        "CLASS": "BST",
    }
    { data:
        { Attributes:
            "DIMENSION_LABELS": ('time', 'beamlet'),
            ...
            "beamlet<ID>_theta": <float>,
            "beamlet<ID>_phi": <float>,
            "beamlet<ID>_coord": <string>,
            "beamlet<ID>_sb": <int>,
            "beamlet<ID>_rcus": [<int>] OR "all" OR null,
            ...
            "bitmode": <int>,
            "hbaelements": <string> OR null,
            "integration": <int>,
            "pol": <string>,
            "rawfile": <string>,
            "rcumode": <int> or [<int>],
            "special": <string> OR null,
            "station": <string>,
            "timestamp": <string>
        {
    }
}
\end{verbatim}

\noindent SST follows the following layout:

\begin{verbatim}
{
    { Attributes:
        "CLASS": "SST",
    }
    { data:
        { Attributes:
            "DIMENSION_LABELS": ('time', 'subband'),
            "hbaelements": <string> OR null,
            "integration": <int>,
            "rawfile": <string>,
            "rcu": <int>,
            "rcumode": <int> or [<int>],
            "special": <string> OR null,
            "station": <string>,
            "timestamp": <string>
        {
    }
}
\end{verbatim}

\noindent XST follows the following layout:

\begin{verbatim}
{
    { Attributes:
        "CLASS": "XST",
    }
    { data:
        { Attributes:
            "DIMENSION_LABELS": ('time', 'subband', 'antpol1', 'antpol2'),
            "hbaelements": <string> OR null,
            "integration": <int>,
            "rawfile": <string>,
            "rcumode": <int> or [<int>],
            "special": <string> OR null,
            "station": <string>,
            "subband": <int>,
            "timestamp": <string>
        {
    }
}
\end{verbatim}

\section{\texttt{issformat}}

The \texttt{issformat} module provides a python interface to meatdata files. A
class is defined for each file type, and functions are provided to read and
write from the define JSON and HDF5 files. 

\subsection{Install}

The code is in a continued state of development. There is support for Python 2
and 3. There are minimum prerequisites, support for HDF5 with h5py is not
required, but recommended, for details see the repository:

\begin{verbatim}
https://github.com/griffinfoster/issformat
\end{verbatim}

\noindent The latest release version can be install via pip:

\begin{verbatim}
pip install issformat
\end{verbatim}

\subsection{Classes}

There are four classes (\texttt{ACC(), BST(), SST(), XST()}), all of which
inherit from a common class \texttt{statData()}. Each class instance contains
data file metadata. There are a common set of functions:

\begin{itemize}
    \item \texttt{setStation()}, \texttt{setRCUmode()}, \texttt{setTimestamp()},
    \texttt{setHBAelements()}, \texttt{setSpecial()}, \texttt{setRawFile()},
    \texttt{setIntegration()}: set metadata variable.
    \item \texttt{setArrayProp()}: set the number of antennas and polarizations,
    useful if using a non-standard station such as KAIRA.
    \item \texttt{printMeta()}: print the current metadata variable
    \item \texttt{writeJSON()}, \texttt{writeHDF5()}: write instance to file.
\end{itemize}

\noindent And, class specific functions:

\begin{itemize}
    \item \textbf{BST} \texttt{setBitmode()}, \texttt{setPol()},
    \texttt{setBeamlet()}: set metadata variable.
    \item \textbf{SST} \texttt{setRCU()}: set metadata variable.
    \item \textbf{XST} \texttt{setSubband()}: set metadata variable.
\end{itemize}

\subsection{Functions}

In addition to the class definitions there are a set of useful functions in
\texttt{issformat}:

\begin{itemize}
    \item \texttt{readJSON()}, \texttt{readHDF5()}, \texttt{read()}: read in
    JSON and HDF5 files, returns a class instance. \texttt{read()} is a general
    wrapper around the two other functions.
    \item \texttt{acc2npy()}, \texttt{bst2npy()}, \texttt{sst2npy()},
    \texttt{xst2npy()}: read in a binary file and return a formatted numpy
    array.
    \item \texttt{npy2acc()}, \texttt{npy2bst()}, \texttt{npy2sst()},
    \texttt{npy2xst()}: write a binary file from a numpy array.
    \item \texttt{printHBAtile()}: print the layout of active elements in an HBA
    tile based on the input hexadecimal string.
\end{itemize}

\subsection{Usage}

The command line interface to \texttt{issformat} is the \texttt{issConverter.py}
script. For in-line documentation use the help flag:

\begin{verbatim}
issConverter.py -h
\end{verbatim}

\subsubsection{Total Correlation (ACC)}

\begin{verbatim}
issConverter.py --rawfile=20120611_124534_acc_512x192x192.dat --standard
--rcumode=3 --station=UK608 -o json --oprefix=20120611_124534_acc_512x192x192
\end{verbatim}

\subsubsection{Beamlet Statistics (BST)}

\begin{verbatim}
issConverter.py --rawfile=20170217_111340_bst_00X.dat --standard
--rcumode=3 --station=UK608 --beamlet=beamlets1.csv
-o json --oprefix=20170217_111340_bst_00X
\end{verbatim}

Since the beamlets require a significant amount of input it is not possible to
enter them directly on the command line. Instead a text file must be used. The
format is a header (\#...) followed by a line for each beamlet:

\begin{verbatim}
# BID THETA PHI COORD SB RCUS
<int> <float> <float> <string> <int> <string/ints>
\end{verbatim}

\noindent an example of the beamlet file is:

\begin{verbatim}
# BID THETA PHI COORD SB RCUS
0 -0.381997 0.831966 AZELGEO 114 all
1 -0.297535 0.178252 AZELGEO 148 all
2 1.094793 1.829449 AZELGEO 259 0,1,2,4,5,6,7,8,9,11,13,14,15,16,17,...
...
\end{verbatim}

\subsubsection{Subband Statistics (SST)}

\begin{verbatim}
issConverter.py --rawfile=20140430_153356_sst_rcu024.dat --standard
--rcumode=3 --station=KAIRA -o json --oprefix=20140430_153356_sst_rcu024
\end{verbatim}


\subsubsection{Subband Correlation (XST)}

\begin{verbatim}
issConverter.py --rawfile=20170728_184348_sb180_xst.dat --rcumode=3
--standard --station=IE613 -o json --oprefix=20170728_184348_sb180_xst
\end{verbatim}

\end{document}             % End of document.

